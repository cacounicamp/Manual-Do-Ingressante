% Este arquivo tex vai ser incluído no arquivo tex principal, não pe preciso
% declarar nenhum cabeçalho

\section{Bem-vindo a essa nova etapa!}

\begin{wrapfigure}{l}{.5\textwidth}
    \includegraphics[width=.5\textwidth]{img/ola_mundo/cb.jpg}
\end{wrapfigure}

Caro colega,

Parabéns, você acabou de ingressar em um dos poucos programas de pós-graduação em Ciência da Computação ou Engenharia Elétrica classificados pela CAPES como curso de excelência máxima (nota 7)! Em breve você irá perceber que não é apenas a estrutura e o nível do corpo docente que fazem uma pós-graduação desse nível, mas também a qualidade e dedicação dos seus alunos. Mas não se assuste! Apesar de todo o suor que você terá pela frente, você verá que ainda é possível conciliar trabalho firme nas aulas e na sua pesquisa com a vida social em Barão Geraldo e Campinas.

Este manual foi organizado pelo CACo (Centro Acadêmico da Computação) em conjunto com a APOGEEU (Associação de Pós-Graduandos da Faculdade de Engenharia Elétrica e de Computação da Unicamp), escrito por diversos alunos que contribuíram com informações que serão especialmente úteis para você, ingressante!

Onde comer? Onde estudar? Onde morar? Tudo isso são dúvidas comuns, que aqui tentamos ajudar a resolver. Não há respostas prontas, cada um tem suas preferências, mas a gente dá uma mão.

O que é APG? E CPG? E CCPG? E RPG? Como eu faço para pegar uma bolsa? A gente também tenta responder todas essas perguntas. E também damos algumas dicas de onde comprar coisas, onde se divertir e alguns telefones úteis.

Já demos os parabéns, então agora vamos para o que importa: dar umas dicas para os seus primeiros passos na pós-graduação da Unicamp e apresentar o CACo e a APOGEEU!